\documentclass[12pt,a4paper,utf8]{ctexart}
\usepackage{graphicx}
\usepackage{amsmath}
\usepackage{amssymb}
\usepackage{subfig}
\usepackage{cite}
\usepackage[ntheorem]{empheq}
\usepackage{enumitem}
\usepackage{fullpage}
\usepackage{cleveref}
\usepackage{cellspace}
\usepackage{listings}
\usepackage{color}
\usepackage{epstopdf}
\usepackage{CJK}
\usepackage{fancyhdr}
\definecolor{gray}{rgb}{0.5,0.5,0.5}
\definecolor{dkgreen}{rgb}{.068,.578,.068}
\definecolor{dkpurple}{rgb}{.320,.064,.680}
        
% set Matlab styles
\lstset{
   language=Matlab,
   keywords={break,case,catch,continue,else,elseif,end,for,function,
      global,if,otherwise,persistent,return,switch,try,while},
   basicstyle=\ttfamily,
   keywordstyle=\color{blue}\bfseries,
   commentstyle=\color{dkgreen},
   stringstyle=\color{dkpurple},
   backgroundcolor=\color{white},
   tabsize=4,
   showspaces=false,
   showstringspaces=false
}

\begin{document}
\CJKfamily{zhkai}	


\begin{center}
\textbf{作业一}\\
\textbf{姓名:马宇骁~~~学号:PB19151769~~日期:2021.5.4}\\
\end{center}

\begin{center}
\fbox{
\begin{minipage}{40em}
\vspace{5cm}
\hspace{20cm}
\end{minipage}}
\end{center}
\vspace{1cm}



\begin{enumerate}
\item[第一题] \textbf{本题考虑使用有限差分方法(finite difference method)解决两点边值问题(boundaryvalue problem)}  

\textsc{Matlab}程序显示如下:
\begin{lstlisting}[frame=single]
clear,clc
% 第一题
A = diag(repmat([2],1,10))+diag(repmat([-1],1,9),1)
+diag(repmat([-1],1,9),-1);
b = [2 -2 2 -1 0 0 1 -2 2 -2];
b = b';
xexact = [1 0 1 0 0 0 0 -1 0 -1];
xexact = xexact';
esp = 1e-15;

% 1
function jacobi(A,b,xexact,esp)
% 其中, A 为线性方程组的系数矩阵,b 为常数项,eps为精度要求
x1 = zeros(10,1);
i = 0;
D = diag(diag(A));
while norm(x1 - xexact,inf) > esp
    i = i+1;
    x1 = D\((D-A)*x1+b);
    document(i,1) = i;
    document(i,2) = norm(x1 - xexact,inf);
end
semilogy(document(:,1),document(:,2),'DisplayName',
'Jacobi迭代法');
%记录横轴纵轴的数据画图
xlabel('迭代次数');
ylabel('误差大小');
end

function GS(A,b,xexact,esp)
x2 = zeros(10,1);
i = 0;
D = diag(diag(A));
L = tril(A,-1);
U = triu(A,1);
while norm(x2 - xexact,inf) > esp
    i = i+1;
    x2 = (D+L)\(b-U*x2);
    document(i,1) = i;
    document(i,2) = norm(x2 - xexact,inf);
end
semilogy(document(:,1),document(:,2),'DisplayName',
'Gauss-Seidel迭代法');
%记录横轴纵轴的数据画图
end

% 2
function SOR(A,b,xexact,esp,w)
x3 = zeros(10,1);
i = 0;
D = diag(diag(A));
L = tril(A,-1);
U = triu(A,1);
I = eye(10);
while (norm(x3 - xexact,inf) > esp) && (i<2500)
    i = i+1;
    x3 = (I+D\L*w)\((I-w*(D\U+I))*x3+D\b*w);
    document(i,1) = i;
    document(i,2) = norm(x3 - xexact,inf);
end
%记录横轴纵轴的数据画图
xlabel('迭代次数');
ylabel('误差大小');
name1 = ['SOR w=',num2str(w)];      %数字字母拼接
semilogy(document(:,1),document(:,2),'DisplayName',name1);
name2 = [' \leftarrow ',name1];     %w的数字指向线
text(75,document(75,2),name2);
legend
end

% 3
function newjacobi(A,b,xexact,esp)
% 其中, A 为线性方程组的系数矩阵,b 为常数项,eps为精度要求
x0 = zeros(10,1);
i = 0;
x = x0;
while norm(x - xexact,inf) > esp
    i = i+1;
    x0 = x;
    x(1,1) = (b(1)-(A(1,2)*x0(2,1)))/A(1,1);
    for j = 2:9
        x(j,1) = (b(j)-(A(j,j-1)*x0(j-1,1) + 
        A(j,j+1)*x0(j+1,1)))/A(j,j);
    end
    x(10,1) = (b(10)-(A(10,9)*x0(9,1)))/A(10,10);
    document(i,1) = i;
    document(i,2) = norm(x - xexact,inf);
end
semilogy(document(:,1),document(:,2),'DisplayName',
'新Jacobi迭代法');
%记录横轴纵轴的数据画图
xlabel('迭代次数');
ylabel('误差大小');
end

function xgs = newGS(A,b,xexact,esp)
x0 = zeros(10,1);
i = 0;
x = x0;
while norm(x - xexact,inf) > esp
    i = i+1;
    x0 = x;
    x(1,1) = (b(1)-(A(1,2)*x0(2,1)))/A(1,1);
    for j = 2:9
        x(j,1) = (b(j)-(A(j,j-1)*x(j-1,1) + 
        A(j,j+1)*x0(j+1,1)))/A(j,j);
    end
    x(10,1) = (b(10)-(A(10,9)*x(9,1)))/A(10,10);
    document(i,1) = i;
    document(i,2) = norm(x - xexact,inf);
end
xgs = x;
end

function newSOR(A,b,xexact,esp,w)
xGS = newGS(A,b,xexact,esp);
x0 = zeros(10,1);
i = 0;
x = x0;
while norm(x - xexact,inf) > esp && (i<2500)
    i = i+1;
    x0 = x;
    for j = 1:10
        x(j,1) = w*xGS(j,1) + (1-w)*x0(j,1); 
    end
end
end
\end{lstlisting}

\textsc{Matlab}第一题最终演示代码显示如下:
\begin{lstlisting}[frame=single]

jacobi(A,b,xexact,esp);
hold on
GS(A,b,xexact,esp);
w = 0.5;
while w < 2
    SOR(A,b,xexact,esp,w);
    w = w+0.5;
end
w = 1.618;
SOR(A,b,xexact,esp,w);

jacobi(A,b,xexact,esp);
fprintf('jacobi改进前的时间');
tic
for i = 1:10
    jacobi(A,b,xexact,esp);
end
toc
newjacobi(A,b,xexact,esp);
fprintf('jacobi改进后的时间');
tic
for i = 1:10
    newjacobi(A,b,xexact,esp);
end
toc

GS(A,b,xexact,esp)
fprintf('GS改进前的时间');
tic
for i = 1:10
    GS(A,b,xexact,esp);
end
toc
newGS(A,b,xexact,esp);
fprintf('GS改进后的时间');
tic
for i = 1:10
    newGS(A,b,xexact,esp);
end
toc

w = 0.5;
while w<2
    SOR(A,b,xexact,esp,w);
    fprintf('SOR改进前的时间');
    tic
    for i = 1:10
        SOR(A,b,xexact,esp,w);
    end
    toc
    newSOR(A,b,xexact,esp,w);
    fprintf('SOR改进后的时间');
    tic
    for i = 1:10
        newSOR(A,b,xexact,esp,w);
    end
    toc
    w = w+0.4;
end
\end{lstlisting}

\begin{figure}  \centering  \includegraphics[height=10cm,width=14cm]{第一张图.eps}  \caption{yes}  \label{1}  \end{figure}  

\textsc{Matlab}第一题最终结果显示如下:
\begin{lstlisting}[frame=single]
jacobi改进前的时间历时 0.369507 秒。
jacobi改进后的时间历时 0.229690 秒。
GS改进前的时间历时 0.017519 秒。
GS改进后的时间历时 0.008679 秒。
SOR w=0.5改进前的时间历时 0.658485 秒。
SOR w=0.5改进后的时间历时 0.006349 秒。
SOR w=1改进前的时间历时 0.527591 秒。
SOR w=1改进后的时间历时 0.003656 秒。
SOR w=1.5改进前的时间历时 0.577475 秒。
SOR w=1.5改进后的时间历时 0.003579 秒。
SOR w=1.618改进前的时间历时 0.583528 秒。
SOR w=1.618改进后的时间历时 0.004871 秒。
>> 
\end{lstlisting}
\textcolor[rgb]{.3,.3,.7}
{\bfseries 从图像当中也可以看出1.618左右时的SOR收敛速度快于其他的$\omega$的
值的收敛速度。}

~\\
~\\
\newpage

\item[第二题]
\textbf{本题将利用求解方程x3−3x2+ 2 = 0的根来深入我们关于Newton方法的收敛速度的讨论。} \\
\textsc{Matlab}程序显示如下:
\begin{lstlisting}[frame=single]
clear , clc

%[-3,0]、[0,2]、[2,4]
NewTon(-1.7);
i = 0.1;
while i<2
    NewTon(i);
    i=i+.5;
end
NewTon(2.5);

function NewTon(a)
syms x;
% 方程为f(x) = 0
f(x) = x^3 - 3*x^2 + 2;
%这是求迭代次数与迭代值
df(x) = diff(f(x),1);
h(x) = f(x)/df(x);
x0 = a;
%这是求迭代次数与迭代值
for i =1:100
    x1 = x0 - h(x0);
    if norm(x1-x0)<1e-15
        document(i,1) = i;
        document(i,2) = x1;
        break;
    else
        document(i,1) = i;
        document(i,2) = x1;
        x0 = x1;
    end
    fprintf('%d     %.15f\n',document(i,1),document(i,2));
end
g(x) = x - f(x)/df(x); 
for p = 1:5          
    g(x) = diff(g(x));
    t = double(g(x1));       %转成浮点数
    if roundn(t,-15) ~= 0       %保留小数点后15位
        break;
    end
end
fprintf('阶数估计p = %d\n',p);
fprintf('\n');
end
\end{lstlisting}
\textsc{Matlab}第二题最终结果显示如下:
\begin{lstlisting}[frame=single]
1     -1.086168521462639
2     -0.805676955223314
3     -0.736322130064943
4     -0.732066517116726
5     -0.732050807782599
6     -0.732050807568877
阶数估计p = 2

1     3.557894736842105
2     3.012915484777833
3     2.781661532896955
4     2.734048844457692
5     2.732054255592772
6     2.732050807579173
7     2.732050807568877
阶数估计p = 2

1     1.050793650793651
2     0.999912409107207
3     1.000000000000448
4     1.000000000000000
阶数估计p = 3

1     0.999326599326599
2     1.000000000203577
3     1.000000000000000
阶数估计p = 3

1     0.775000000000000
2     1.007998683344306
3     0.999999658813342
4     1.000000000000000
阶数估计p = 3

1     2.800000000000000
2     2.735714285714286
3     2.732062373480407
4     2.732050807684724
5     2.732050807568877
阶数估计p = 2
\end{lstlisting}
\textcolor[rgb]{.3,.3,.7}
{\bfseries \quad(c)关于c小问,通过第二问简单粗暴的整型判断,发现在判定1附近的根时最
后的收敛阶数出现大于2的情况。根据上课所听讲的内容:电脑({MATLAB})在处理求根问题的时候,
若当前点距离精确值较远会采取“作弊”,即调用比Newton法局部更快速收敛的方法进行优化,从
而使得可能在局部出现比二阶收敛更快的现象。而且牛顿法的二阶也只是整体近似,局部出现微小偏
差也情有可原。}


\newpage
\item[第三题]
\textbf{我们已经学习了使用幂法求解特征值问题。} \\
{\bfseries \quad(a)关于a小问,算法思想伪代码如下:}
根据课堂中的思想改进后的算法:\\
\begin{equation}
q^{old} = (1,1,\ldots,1)^T \\
\end{equation}
\begin{equation}
\hat{q}^{old} = q^{old} / \|q^{old}\|_\infty\\
m,\varepsilon\\
\end{equation}
\begin{equation}
for~ k = 1:m\\
\end{equation}
\begin{equation}
\quad q^{new} = A\hat{q}^{old}\\
\end{equation}
\begin{equation}
\quad \lambda =  \left\|q^{new}\right\|_{\infty} \\
\end{equation}
\begin{equation}
\quad \hat{q}^{new} = q^{new} / \lambda\\
\end{equation}
\begin{equation}
\quad if( \left\|\hat{q}^{old} - \hat{q}^{new} \right\|_\infty < \varepsilon)\\
\end{equation}
\begin{equation}
\quad \hat{q}^{old} = \hat{q}^{new}\\
\end{equation}
\begin{equation}
\quad \quad return ~\lambda,hat{q}^{new},break\\
\end{equation}
\begin{equation}
\quad elseif( \left\|\hat{q}^{old} + \hat{q}^{new} \right\|_\infty  < \varepsilon)\\
\end{equation}
\begin{equation}
\quad \quad return~ -\lambda,hat{q}^{new},break\\
\end{equation}
\begin{equation}
\quad \hat{q}^{old} = \hat{q}^{new}\\
\end{equation}
\begin{equation}
\quad elseif( \left\|\hat{q}^{old} - \hat{q}^{newnew} \right\|_\infty < \varepsilon)\\
\end{equation}
\begin{equation}
\quad \quad return~ \lambda^2,hat{q}^{newnew},break\\
\end{equation}
\begin{equation}
\quad \hat{q}^{old} = \hat{q}^{new}\\
\end{equation}

end\\
end\\


\textsc{Matlab}程序显示如下:
\begin{lstlisting}[frame=single]
clear, clc

A = [-148 -105 -83 -67;
    488 343 269 216;
    -382 -268 -210 -170;
    50 38 32 29]
power(A);

B = [222 580 584 786;
    -82 -211 -208 -288;
    37 98 101 132;
    -30 -82 -88 -109]

power(B);

function power(A) 
lam2 = -1;
% lam2最后通过判断是否存在两个最大特征值
%幂法法二
qold = ones(4,1);
q_old = qold/norm(qold,inf);
% 归一化,除以无穷范数
for j = 1:1000
    qnew = A * q_old;
    lam = norm(qnew,inf);
    % λ作为特征值
    q_new = qnew/lam;
    %归一化
    % 出现俩绝对值相等的最大特征值时:
    qnewn = A * A * q_old;
    qnn = A * q_new;
    q_nn = qnn/norm(qnn,inf);
    if norm(q_old - q_new,inf) < 1e-15        
        %特征值为正时
        break;
    elseif norm(q_old + q_new,inf) < 1e-15    
        %特征值为负时
        lambda = -lambda;
        break;
    elseif norm(q_old - q_nn,inf) < 1e-15  
        %不满足前两个条件,返回特征值的平方
        lam2 = norm(qnewn,inf) / norm(q_old,inf);
		%得到特征值的平方
        break;
    end
    q_old = q_new;
end
fprintf('迭代次数:%d\n',j);

if lam2 > 0     
    %俩绝对值相等的最大特征值时
    lam_1 = sqrt(lam2);     %正特征值
    lam_2 = -lam_1;         %负特征值
    fprintf('模最大的特征值:\n');
    fprintf('%.15f\n',lam_1);
    fprintf('%.15f\n',lam_2);
    fprintf('模最大的特征向量\n');
    disp(q_nn);
    disp(q_new);
else
    %只有一个绝对值最大时
    fprintf('模最大的特征值:');
    fprintf('%.16f\n',lam);
    fprintf('模最大的特征向量:\n');
    disp(q_new);
end
end
\end{lstlisting}

\textsc{Matlab}第三题前三问最终结果显示如下:
\begin{lstlisting}[frame=single]
A =

  -148  -105   -83   -67
   488   343   269   216
  -382  -268  -210  -170
    50    38    32    29

迭代次数:41
模最大的特征值:8.0000000000002132
模最大的特征向量:
   -0.3103
    1.0000
   -0.7931
    0.1379

B =

   222   580   584   786
   -82  -211  -208  -288
    37    98   101   132
   -30   -82   -88  -109

迭代次数:85
模最大的特征值:
4.999999999999742
-4.999999999999742
模最大的特征向量
   -1.0000
    0.2857
   -0.1786
    0.2143

    1.0000
   -0.3636
    0.1591
   -0.1364

>> 
\end{lstlisting}

\textsc{Matlab}第三题最后一问由反幂法和平移最终结果显示如下:
\begin{lstlisting}[frame=single]
clear,clc

p = 0.8-0.6*1i;
rng(2);
A = 2*rand(100)-1;

InversePower(A,p);

function InversePower(A,p)
I = eye(100,100);
u0 = ones(100,1);
v = (A - p * I) \ u0;
u = v / norm(v, inf);
i = 0;
while norm(u - u0, inf)>1e-15 && i<10000
    u0 = u;
    v = (A - p * I) \ u0;
    u = v / norm(v, inf);
    i = i + 1;
end
fprintf('迭代次数:%d\n',i);
fprintf('特征向量:');
disp(u);
lam = p + 1/norm(v,inf);
fprintf('最接近的特征值:%d\n',lam);
end
\end{lstlisting}




\end{enumerate}




\end{document}
